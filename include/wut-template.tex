
% !TeX TS-program = pdflatexmk
% !BIB TS-program = biber

%-----------------------------------------------
%  Engineer's & Master's Thesis Template
%  Copyleft by Artur M. Brodzki & Piotr Woźniak
%  Warsaw University of Technology, 2019-2021
%-----------------------------------------------

\documentclass[
    bindingoffset=5mm,  % Binding offset
    footnoteindent=3mm, % Footnote indent
    hyphenation=true    % Hyphenation turn on/off
]{include/wut-thesis}


%-------------------------------------------------------------
% Wybór wydziału:
%  \facultyeiti: Wydział Elektroniki i Technik Informacyjnych
%  \facultymeil: Wydział Mechaniczny Energetyki i Lotnictwa
% --
% Rodzaj pracy: \EngineerThesis, \MasterThesis
% --
% Wybór języka: \langpol, \langeng
%-------------------------------------------------------------
\facultyeiti    % Wydział Elektroniki i Technik Informacyjnych
\EngineerThesis % Praca inżynierska
\langpol % Praca w języku polskim
\usepackage{float}

$if(lhs)$
\lstnewenvironment{code}{\lstset{language=Haskell,basicstyle=\small\ttfamily}}{}
$endif$
$if(highlighting-macros)$
$highlighting-macros$
\usepackage{fvextra}
\DefineVerbatimEnvironment{Highlighting}{Verbatim}{breaklines,fontsize=$if(code-block-font-size)$$code-block-font-size$$else$\small$endif$,commandchars=\\\{\}}
$endif$
\usepackage{graphicx}
\usepackage[backend=biber]{biblatex}

$if(csl-refs)$
\newlength{\cslhangindent}
\setlength{\cslhangindent}{1.5em}
\newenvironment{cslreferences}%
  {$if(csl-hanging-indent)$\setlength{\parindent}{0pt}%
  \everypar{\setlength{\hangindent}{\cslhangindent}}\ignorespaces$endif$}%
  {\par}
$endif$

$for(bibliography)$
\addbibresource{$bibliography$}
$endfor$

% Redefine \includegraphics so that, unless explicit options are
% given, the image width will not exceed the width or the height of the page.
% Images get their normal width if they fit onto the page, but
% are scaled down if they would overflow the margins.
\makeatletter
\def\ScaleWidthIfNeeded{%
 \ifdim\Gin@nat@width>\linewidth
    0.9\linewidth
  \else
    \Gin@nat@width
  \fi
}
\def\ScaleHeightIfNeeded{%
  \ifdim\Gin@nat@height>0.9\textheight
    0.9\textheight
  \else
    \Gin@nat@width
  \fi
}
\makeatother



\begin{document}

%------------------
% Strona tytułowa
%------------------
\instytut{$institute$}
\kierunek{$subject$}
\specjalnosc{$specialty$}
\title{
    $title-pl$
}
% Tytuł po angielsku do angielskiego streszczenia
% In english theses, you may remove this command
\engtitle{
    $title-en$
}
\author{$author$}
\album{$album$}
\promotor{$promotor$}
\date{\the\year}
\maketitle

%-------------------------
% Streszczenie po polsku
%-------------------------
\cleardoublepage % Zaczynamy od nieparzystej strony
\streszczenie $abstract-pl$
\slowakluczowe $keywords-pl$

%----------------------------
% Streszczenie po angielsku
%----------------------------
\clearpage
\abstract $abstract-en$
\keywords $keywords-en$

%----------------------------
% Oświadczenie o autorstwie
%----------------------------
\cleardoublepage % Zaczynamy od nieparzystej strony
\pagestyle{plain}
\makeauthorship

%--------------
% Spis treści
%--------------
\cleardoublepage % Zaczynamy od nieparzystej strony
\tableofcontents

%------------
% Rozdziały
%------------
\cleardoublepage % Zaczynamy od nieparzystej strony
\pagestyle{headings}

 % Można też pisać rozdziały w jednym pliku.
\clearpage % Zawsze zaczynamy rozdział od nowej strony

\floatplacement{figure}{H}

\setkeys{Gin}{width=\ScaleWidthIfNeeded,height=\ScaleHeightIfNeeded,keepaspectratio}%

$body$

%---------------
% Bibliografia
%---------------

\cleardoublepage % Zaczynamy od nieparzystej strony

\nocite{*}
\printbibliography

%--------------------------------------
% Spisy: rysunków, tabel, załączników
%--------------------------------------
\clearpage
\pagestyle{plain}

\listoffigurestoc    % Spis rysunków.
\vspace{1cm}         % vertical space

\listoftablestoc     % Spis tabel.
\vspace{1cm}         % vertical space

$if(enableappendices)$
\listofappendicestoc % Spis załączników
$endif$
% Wykaz symboli i skrótów.
% Pamiętaj, żeby posortować symbole alfabetycznie
% we własnym zakresie. Makro \acronymlist
% generuje właściwy tytuł sekcji, w zależności od języka.
% Makro \acronym dodaje skrót/symbol do listy,
% zapewniając podstawowe formatowanie.
\vspace{0.8cm}
\acronymlist
${ for(acronyms) }
\acronym{${ it.acronym }}{${ it.expansion }}
${ endfor }
%-------------
% Załączniki
%-------------

% Obrazki i tabele w załącznikach nie trafiają do spisów
\captionsetup[figure]{list=no}
\captionsetup[table]{list=no}

% Używając powyższych spisów jako szablonu,
% możesz dodać również swój własny wykaz,
% np. spis algorytmów.

\end{document} % Dobranoc.
